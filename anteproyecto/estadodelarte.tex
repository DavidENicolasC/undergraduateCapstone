\section{Estado del arte}

\begin{table}[h]
	\caption{Estado del arte}
	%\centering
	\begin{tabular}{p{5cm}p{4cm}p{3.6cm}p{4cm}}
		\textbf{Título} & \textbf{Autores} & \textbf{Tipo de publicación, lugar y fecha} & \textbf{Descripción} \\ 
		\midrule
		FIKA: A Conformal Geometric Algebra Approach to a Fast Inverse Kinematics Algorithm for an Anthropomorphic Robotic Arm \newline\newline
		FIKA: Un enfoque de álgebra geométrica conforme para un eficaz algoritmo cinemático inverso para un brazo robótico antropomórfico &  
		Oscar Carbajal-Espinosa; Leobardo Campos-Macías; Miriam Díaz-Rodríguez \newline\newline
		Instituto Tecnológico y de Estudios Superiores de Monterrey; Intel Corporation; Tecnológico Nacional de México & 
		\begin{center}Artículo \par \includegraphics[width=2cm]{mexico.jpg} \par Mexico \par 2024\end{center} & 
		Propone un método geométrico iterativo de 3 fases para resolver el problema de la cinemática inversa.\newline\newline
		Sin embargo, requiere un tiempo de procesamiento de datos inviable.\\
		\midrule
		Geometric Approach for Inverse Kinematics of the FANUC CRX Collaborative Robot. \newline\newline
		Aproximación geométrica para la cinemática inversa del robot colaborativo FANUC CRX. &  
		Manel Abbes; Gérard Poisson \newline\newline 
		University of Orléans & 
		\begin{center}Artículo \par \includegraphics[width=2cm]{turquia.png} \par Francia \par 2024\end{center} & 
		El artículo propone un método geométrico para resolver la cinemática inversa de este robot.\newline\newline Sin embargo, el algoritmo requiere ajustarse si las longitudes del robot varían. \\
	\end{tabular}
\end{table}

\newpage
\begin{table}[htb]
	\caption{Estado del arte (continuación)}
	\centering
	\begin{tabular}{p{5cm}p{4cm}p{3.6cm}p{4cm}}
		\textbf{Título} & \textbf{Autores} & \textbf{Tipo de publicación, lugar y fecha} & \textbf{Descripción} \\ 
		\midrule
		Inverse kinematics solution and control method of 6-degree-of-freedom manipulator based on deep reinforcement learning. \newline\newline
		Solución de cinemática inversa y método de control de un manipulador de 6 grados de libertad basado en aprendizaje de refuerzo profundo. &  
		Chengyi Zhao; Yimin Wei; Junfeng Xiao; Yong Sun; Dongxing Zhang; Qiuquan Guo; Jun Yang \newline\newline 
		University of Electronic Science and Technology of China & 
		\begin{center}Artículo \par \includegraphics[width=2cm]{china.png} \par China \par 2024\end{center} & 
		Propone un algoritmo de aprendizaje por refuerzo que calcula la distancia entre el efector final y la posición deseada. \newline\newline Sin embargo, requiere un reentrenamiento para ser adaptado a otras longitudes de brazos robóticos. \\
	\end{tabular}
\end{table}