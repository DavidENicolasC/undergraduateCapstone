\section{Estado del arte}

\begin{table}[h]
	\caption{Estado del arte}
	%\centering
	\begin{tabular}{p{5cm}p{4cm}p{3.6cm}p{4cm}}
		\textbf{Título} & \textbf{Autores} & \textbf{Tipo de publicación, lugar y fecha} & \textbf{Descripción} \\ 
		\midrule
		FIKA: A Conformal Geometric Algebra Approach to a Fast Inverse Kinematics Algorithm for an Anthropomorphic Robotic Arm \newline\newline
		FIKA: Un enfoque de álgebra geométrica conforme para un eficaz algoritmo cinemático inverso para un brazo robótico antropomórfico &  
		Oscar Carbajal-Espinosa; Leobardo Campos-Macías; Miriam Díaz-Rodríguez \newline\newline
		Instituto Tecnológico y de Estudios Superiores de Monterrey; Intel Corporation; Tecnológico Nacional de México & 
		\begin{center}Artículo \par \includegraphics[width=2cm]{mexico.jpg} \par Mexico \par 2024\end{center} & 
		Propone un método geométrico iterativo de 3 fases para resolver el problema de la cinemática inversa.\newline\newline
		Sin embargo, requiere un tiempo de procesamiento de datos inviable para el tiempo de respuesta del sistema requerido por SEDPC.\\
		\midrule
		Implementation of singularity-free inverse kinematics for humanoid robotic arm using Bayesian optimized deep neural network. \newline\newline
		Implementación de cinemática inversa sin singularidad para brazo robótico humanoide utilizando una red neuronal profunda optimizada por métodos Bayesianos. &  
		Omur Aydogmus; Gullu Boztas \newline\newline 
		Firat University & 
		\begin{center}Artículo \par \includegraphics[width=2cm]{turquia.png} \par Turquía \par 2024\end{center} & 
		Utiliza una red neuronal basada en aprendizaje profundo para resolver la cinemática inversa en una simulación.\newline\newline Sin embargo, el proyecto se enfocó en . \\
	\end{tabular}
\end{table}

\newpage
\begin{table}[htb]
	\caption{Estado del arte (continuación)}
	\centering
	\begin{tabular}{p{3.8cm}p{3.8cm}p{3.8cm}p{3.8cm}}
		\textbf{Título} & \textbf{Autores} & \textbf{Tipo de publicación, lugar y fecha} & \textbf{Descripción} \\ 
		\midrule
		Inverse kinematics solution and control method of 6-degree-of-freedom manipulator based on deep reinforcement learning. \newline\newline
		Solución de cinemática inversa y método de control de un manipulador de 6 grados de libertad basado en aprendizaje de refuerzo profundo. &  
		Chengyi Zhao; Yimin Wei; Junfeng Xiao; Yong Sun; Dongxing Zhang; Qiuquan Guo; Jun Yang \newline\newline 
		University of Electronic Science and Technology of China & 
		\begin{center}Artículo \par \includegraphics[width=2cm]{china.png} \par China \par 2024\end{center} & 
		Propone un algoritmo de aprendizaje por refuerzo que calcula la distancia entre el efector final y la posición deseada. \newline\newline Sin embargo, se volverá ineficaz cuando se adapte a brazos de diferente longitud. \\
	\end{tabular}
\end{table}