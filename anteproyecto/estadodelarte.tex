\section{Estado del arte}

\begin{table}[h]
	\caption{Estado del arte}
	%\centering
	\begin{tabular}{p{5cm}p{4cm}p{3.6cm}p{4cm}}
		\textbf{Título} & \textbf{Autores} & \textbf{Tipo de publicación, lugar y fecha} & \textbf{Descripción} \\ 
		\midrule
		FIKA: A Conformal Geometric Algebra Approach to a Fast Inverse Kinematics Algorithm for an Anthropomorphic Robotic Arm \cite{estado1} \newline\newline
		FIKA: Un enfoque de álgebra geométrica conforme para un eficaz algoritmo cinemático inverso para un brazo robótico antropomórfico &  
		Oscar Carbajal-Espinosa; Leobardo Campos-Macías; Miriam Díaz-Rodríguez \newline\newline
		Instituto Tecnológico y de Estudios Superiores de Monterrey; Intel Corporation; Tecnológico Nacional de México & 
		\begin{center}Artículo \par \includegraphics[width=2cm]{mexico.jpg} \par Mexico \par 2024\end{center} & 
		Propone un método geométrico iterativo de 3 fases para resolver el problema de la cinemática inversa.\newline\newline
		Sin embargo, requiere un tiempo de procesamiento de datos inviable.\\
		\midrule
		SK-PSO: A Particle Swarm Optimization Framework with SOM and K-Means for Inverse Kinematics of Manipulators \cite{estado2} \newline\newline
		SK-PSO: Un marco de optimización de enjambre de partículas con SOM y K-Means para la cinemática inversa de manipuladores &  
		Fei Liu; Changqin Gao; Lisha Liu \newline\newline 
		Shenzen Polytechnic University & 
		\begin{center}Artículo \par \includegraphics[width=2cm]{china.png} \par China \par 2024\end{center} & 
		Propone un enfoque híbrido basado en optimización por enjambre de partículas (PSO), mapas autoorganizados (SOM) y K-means.\newline\newline Sin embargo, la clusterización por K-means requiere conocer el número exacto de clústeres de antemano. \\
	\end{tabular}
\end{table}

\newpage
\begin{table}[htb]
	\caption{Estado del arte (continuación)}
	\centering
	\begin{tabular}{p{5cm}p{4cm}p{3.6cm}p{4cm}}
		\textbf{Título} & \textbf{Autores} & \textbf{Tipo de publicación, lugar y fecha} & \textbf{Descripción} \\ 
		\midrule
		Inverse kinematics analysis of a wrist rehabilitation robot using artificial neural network and adaptive Neuro-Fuzzy inference system \cite{estado3} \newline\newline
		Análisis cinemático inverso de un robot de rehabilitación de muñeca mediante red neuronal artificial y sistema de inferencia adaptativo neuro-difuso &  
		Behzad Saeedi; Majid Mohammadi Moghaddam; Majid Sadedel \newline\newline 
		Tarbiat Modares University & 
		\begin{center}Artículo \par \includegraphics[width=2cm]{iran.jpg} \par Irán \par 2024\end{center} & 
		Utiliza un sistema de inferencia neuro-difuso adaptativo para inferir la mejor aproximación a las posiciones. \newline\newline Sin embargo, la red neuro-difusa crece exponencialmente conforme aumenta la precisión de los datos de entrenamiento. \\
	\end{tabular}
\end{table}