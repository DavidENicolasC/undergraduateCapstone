\section{Objetivo general}

Desarrollar e implementar un sistema de control utilizando una Raspberry Pi y sensores inerciales para controlar la posición del efector final de un brazo robótico de 2 DoF de longitud variable.

\newpage
\section{Objetivos específicos}
\begin{itemize}
	
	\item Desarrollar e implementar un programa en C++, utilizando los ángulos de inclinación en los tres ejes obtenidos por los sensores MPU6050, para determinar la posición del sensor en el extremo de la ortesis de brazo.
	
	\item Diseñar e implementar un modelo de clusterización para determinar los ángulos de la cinemática inversa.
	
	\item Implementar la comunicación entre la Raspberry Pi y el software SettDev por medio de sockets TCP en C++ y C\# para guardar y reproducir las instrucciones en la simulación en 3D de un brazo robótico.
	
	\item Desarrollar e implementar una interfaz gráfica de usuario por medio del framework Qt para visualizar en una pantalla táctil de 7 pulgadas la cinemática inversa del brazo robótico a controlar.
	
	\item Implementar la comunicación entre el software SettDev y el PLC Kaab utilizando el módulo para el control de servomotores instalado en el PLC para enviar los ángulos de inclinación al controlador y reproducir los ángulos de inclinación en los servomotores posicionales.
	
\end{itemize}