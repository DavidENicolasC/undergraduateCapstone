\section{Objetivo general}

Desarrollar e implementar un sistema de control utilizando una Raspberry Pi, sensores inerciales y una red neuro-difusa, para controlar la posición del efector final de un brazo robótico de 2 DoF de longitud variable.

\newpage
\section{Objetivos específicos}
\begin{itemize}
	
	\item Desarrollar e implementar un programa en C++, utilizando los ángulos de inclinación en los tres ejes obtenidos por los sensores MPU6050, para determinar la posición del sensor en el extremo de la ortesis de brazo.
	
	\item Implementar una red neuro-difusa en C++ entrenada con aprendizaje supervisado para resolver el problema de la cinemática inversa.
	
	\item Implementar la comunicación entre la Raspberry Pi y el software SettDev por medio de sockets TCP en C++ y C\# para enviar los ángulos de inclinación calculados al software, para poder guardar los datos y permitir reproducirlos en la simulación en 3D de un brazo robótico.
	
	\item Desarrollar e implementar una interfaz gráfica de usuario utilizando una pantalla táctil por medio del framework Qt para visualizar la cinemática inversa del brazo robótico a controlar.
	
	\item Desarrollar e implementar en SettDev la comunicación entre el módulo del brazo robótico en 3D y el PLC por medio de sockets UDP en C\# para enviar los ángulos de inclinación al controlador y reproducir los ángulos de inclinación en los servomotores posicionales.
	
\end{itemize}