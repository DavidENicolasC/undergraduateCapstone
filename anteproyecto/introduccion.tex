\section{Introducci\'on}

Los brazos robóticos articulados son sistemas mecánicos con articulaciones rotativas, diseñados para replicar funciones del brazo humano, incluyendo movimientos de rotación y alcance. Suelen tener una pieza en el extremo del robot llamado efector final, como se muestra en la Figura \ref{fig:brazoR}, que realiza la función del robot en el entorno (soldar, manipular objetos, etc.). Cada unión en las articulaciones representa un grado de libertad (DoF).

\begin{figure}[htb]
	\centering
	\includegraphics[scale=0.6]{brazo.jpg}
	\caption{Partes de un brazo robótico articulado de 3 DoF}
	\label{fig:brazoR}
\end{figure}

Sistemas Eléctricos de Potencia Computarizada (SEPDC) es una empresa mexicana que se dedica a fabricar la serie Kaab (Fig. 2) de Controladores Lógicos Programables (PLC), computadoras especializadas para la automatización industrial (tienen inmunidad al ruido eléctrico y resistencia a la vibración y al impacto). Cada uno de ellos puede ser operado de forma remota a través de un software llamado SettDev.

\begin{figure}[htb]
	\centering
	\includegraphics[scale=1]{plckaab.png}
	\caption{PLC-Kaab fabricado por SEDPC}
\end{figure}

Dichos PLCs pueden ser empleados para controlar sistemas críticos, como se muestra en la Figura .