\section{Introducci\'on}

Los brazos robóticos articulados son sistemas mecánicos con articulaciones rotativas, diseñados para replicar funciones del brazo humano, a través de movimientos de alcance. Suelen tener una pieza en el extremo del robot llamado efector final, como se muestra en la Figura \ref{fig:brazoR}, que realiza la función del robot en el entorno (soldar, manipular objetos, etc.). Cada eje de movilidad en las articulaciones representa un grado de libertad (GDL); de este modo, si una articulación puede girar de izquierda a derecha, y también de arriba hacia abajo, se dice que tiene dos GDL.

\vspace{1cm}

\begin{figure}[htb]
	\centering
	\includegraphics[scale=0.4]{brazo.jpeg}
	\caption{Partes de un brazo robótico articulado de 2 GDL, sin contar el del efector final}
	\label{fig:brazoR}
\end{figure}

\newpage
Sistemas Eléctricos de Potencia Computarizada (SEPDC) es una empresa mexicana que se dedica a fabricar la serie Kaab de Controladores Lógicos Programables (PLC), computadoras especializadas para la automatización industrial (tienen inmunidad al ruido eléctrico y resistencia a la vibración y al impacto). Cada uno de ellos puede ser operado de forma remota a través de un software llamado SettDev. En la Figura \ref{fig:plc} se muestra el PLC-Kaab.

\begin{figure}[htb]
	\centering
	\includegraphics[scale=0.9]{plckaab.png}
	\caption{PLC-Kaab fabricado por SEDPC}
	\label{fig:plc}
\end{figure}

Los PLC´s son empleados para el control de sistemas, como lo demuestra la Figura \ref{fig:siscritico}. Por medio de una versión del software SettDev instalado en el PLC, se monitorea una planta de agua tratada, determinando los niveles de llenado del tanque.

\begin{figure}[htb]
	\centering
	\includegraphics[scale=0.9]{sistemaplc.png}
	\caption{Uso del PLCKaab para controlar un sistema de llenado de agua}
	\label{fig:siscritico}
\end{figure}