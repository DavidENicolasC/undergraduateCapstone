\subsection{Resultados}

Se encontró que la cantidad total de posiciones alcanzables es de 163, 504 posiciones cuyas coordenadas son enteras.

Los resultados de la prueba descrita se muestran en la siguiente tabla.

\begin{table}[ht]
	\centering
	\begin{tabular}{|c|c|c|c|c|}
		\hline
		\textbf{$L_1$ (cm)} & \textbf{$L_2$ (cm)} & \textbf{Tiempo total (s)} & \textbf{Error máximo (cm)} & \textbf{Tamaño (MB)} \\
		\hline
		7 & 8 & 100 & 0.66 & 3.08 \\
		10.5 & 12 & 110 & 0.65 & 3.12 \\
		18 & 21 & 120 & 0.65 & 3.17 \\
		\hline
	\end{tabular}
	\caption{Resultados}
\end{table}

Se pudo notar que el tiempo de la clusterización es de alrededor de 2 minutos. Esto es por la cantidad de posturas generadas para determinar la más cercana a una posición en particular; si la base rota en 360°, y las articulaciones rotan 90°, entonces:

\begin{equation}
	360 \cdot 90 \cdot 90 = 2,916,000
\end{equation}

Lo que quiere decir que se generan 2,916,000 posturas. Esto es un valor relativamente alto, que puede ser reducido si se varían en ángulos de 2° en las articulaciones; por el enfoque que hemos dado al modelo, sabemos que la distancia de un punto con coordenadas enteras a otro es, por la distancia euclidiana, de 0.707 cm, de modo que la precisión no disminuye significativamente si disminuimos la cantidad de ángulos que pueden ser generados por cada articulación. Sin embargo, la cantidad de posiciones alcanzables puede verse reducida, debido a que algunas posiciones solo podrían ser alcanzadas por alguna postura en particular que no se consideró para la clusterización. Este valor aumentará exponencialmente conforme aumente la cantidad de grados de libertad.

También puede notarse que el tamaño conjunto de los archivos de clusterización no crece exponencialmente al variar la longitud de las articulaciones; se mantiene cercana a 3,2 MB, lo cual es un tamaño relativamente pequeño en comparación con el almacenamiento de la Raspberry Pi utilizado para este proyecto, de 16 GB. Si se aumentan los grados de libertad, cada archivo de clusterización guarda 1 byte más de información.