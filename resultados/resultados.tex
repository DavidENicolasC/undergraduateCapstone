\section{Resultados}

El resultado de la métrica de la desviación de la posición final, tomando el máximo valor con $L = 21 cm$, es de 0,65 cm. Dado que no hay un estándar sobre un umbral de precisión para una determinada aplicación, esto último depende de las especificaciones establecidas en el momento de integrarlo en algún sistema.

La comparación de esta precisión con el estado del arte se muestra en la Tabla \ref{Precision}. Cabe resaltar que los trabajos del estado del arte se aplicaron con ángulos no enteros que no son alcanzables por servomotores posicionales, de modo que, para la aplicación específica del robot, su precisión puede verse sesgada.

\begin{table}[ht]
	\centering
	\begin{tabular}{p{5cm}p{4cm}p{3.6cm}p{4cm}}
		\hline
		\textbf{Sistema implementado (cm)} & \textbf{Método iterativo (cm)} & \textbf{Método por clusterización K-Means (cm)} & \textbf{Método geométrico (cm)} \\
		\hline
		0,65 cm & 0,36 cm & 0,45 & 0,12 cm \\
		\hline
	\end{tabular}
	\caption{Precisión}
	\label{tab:Precision}
\end{table}

Los resultados de la prueba de eficiencia se muestran a continuación.

