\section{Pruebas}

Para medir la precisión del sistema, se aplicó la métrica de la desviación de la posición final de la norma ISO 9283:1998, que calcula la distancia euclidiana entre la posición deseada $(x_1, y_1, z_1)$ y la posición alcanzada $(x_2, y_2, z_2)$, de acuerdo con la ecuación:

\begin{equation}
	d = |\sqrt{(x_2 - x_1)^2 + (y_2 - y_1)^2 + (z_2 - z_1)^2}|
\end{equation}

Y se tomó el máximo valor. 

Para medir la calidad del software, se aplicó la norma ISO 9126. Esta se compone por 7 criterios: Funcionalidad, confiabilidad, usabilidad, eficiencia, mantenibilidad, portabilidad. 

El equipo utilizado para generar las combinaciones de posiciones y combinaciones de ángulos de rotación posee las siguientes características:

\begin{itemize}
	\item Procesador Intel Core i5 de 2,4 GHz de 4 núcleos
	\item 8 GB de RAM
	\item 256 GB de SSD
	\item Sistema operativo Windows 10
\end{itemize}