\section{Pruebas}

Para determinar cómo cambia la precisión del sistema en la generación de los grupos, se variaron los parámetros $L_1, L_2, \alpha_1, \alpha_2, \alpha_3$, para determinar la máxima y mínima distancia euclidiana obtenida, y así obtener la máxima imprecisión para una posición particular. En particular, se variaron $L_1$ y $L_2$ de 0.5 cm a 1 cm, y los saltos de $\alpha_1, \alpha_2, \alpha_3$ variaron en un factor desde 1 hasta 3, sin tomar en cuenta la variación de saltos de acuerdo con el rango de la longitud que se consideró en la etapa de Procesamiento. También se determinó la cantidad de clústeres generados, así como el tamaño máximo y mínimo de los clústeres generados.