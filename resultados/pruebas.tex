\section{Pruebas}

Se midió el tiempo total de ejecución de la clusterización utilizando la biblioteca estándar chrono de C++, que utiliza un reloj basado en el sistema operativo, y se aplicó la métrica del error de posicionamiento euclidiano entre el punto $P(x_i, y_i, z_i)$ para determinar la máxima distancia Euclidiana (y por lo tanto, el error máximo del sistema). Las pruebas se realizaron con los siguientes datos:

\begin{itemize}
	\item $L_1 = 3, L_2 = 4$
	\item $L_1 = 7, L_2 = 8$
	\item $L_1 = 10.5, L_2 = 12$
	\item $L_1 = 18, L_2 = 20$
\end{itemize}

El equipo utilizado para generar la clusterización posee las siguientes características:

\begin{itemize}
	\item Procesador Intel Core i5 de 2,4 GHz de 4 núcleos
	\item 8 GB de RAM
	\item 256 GB de SSD
	\item Sistema operativo Windows 10
\end{itemize}

