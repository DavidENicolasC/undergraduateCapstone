\section{Pruebas}

Se aplicó la métrica del error de posicionamiento euclidiano para determinar la máxima distancia Euclidiana entre las poses asociadas a una posición alcanzable. La distancia euclidiana entre la posición final de dos poses que varían en solo un grado en uno de sus ángulos de rotación (Como $(\alpha_1,\alpha_2,\alpha_3)$ y $(\alpha_1 + 1, \alpha_2, \alpha_3)$), aumenta de forma directamente proporcional a la longitud de los eslabones, de acuerdo con la ecuación de la distancia de la cuerda:

\begin{equation}
	d_{cuerda} = 2 \cdot L \cdot \sin(\frac{\pi}{360}) = 0,017 \cdot L
\end{equation}

Donde $L$ es la longitud del eslabón. Para la maxima longitud $L = 21$ cm, $d_{cuerda} = 0,017 \cdot 21 = 0,36$ cm.

El equipo utilizado para generar la clusterización posee las siguientes características:

\begin{itemize}
	\item Procesador Intel Core i5 de 2,4 GHz de 4 núcleos
	\item 8 GB de RAM
	\item 256 GB de SSD
	\item Sistema operativo Windows 10
\end{itemize}

