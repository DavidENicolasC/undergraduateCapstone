\subsection{Conclusiones}

Finalmente, se concluye que el modelo de clusterización obtenido, pese a requerir una cantidad considerable de tiempo para la generación de datos, posee un tiempo de respuesta considerablemente reducido, lo que no compromete los recursos disponibles de la Raspberry Pi. Además, el tamaño de los archivos de clusterización generados no crece exponencialmente conforme varían las longitudes de las articulaciones, y, conforme aumenten los grados de libertad, el tamaño total aumenta de forma lineal. Sin embargo, incluir más grados de libertad para determinar la posición en el modelo aumenta exponencialmente el tiempo para la generación de datos, lo que requiere 