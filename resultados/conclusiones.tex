\section{Conclusiones}

Finalmente, se concluye que el modelo obtenido, pese a requerir una cantidad considerable de tiempo para la generación de datos, posee un tiempo de ejecución muy reducido, lo que no compromete los recursos disponibles de la Raspberry Pi. Además, el tamaño de los archivos de clusterización generados no crece exponencialmente conforme varían las longitudes de las articulaciones, y, conforme aumenten los grados de libertad, el tamaño total aumenta de forma lineal. Incluir más grados de libertad para determinar la posición en el modelo aumenta exponencialmente el tiempo para la generación de datos, lo que requiere una estrategia de optimización si se quiere llevar a dicha aplicación. La precisión del modelo se establece con una cota máxima desde el principio; conforme se disminuya esta cota, se aumentará la aproximación, lo que implica que el tamaño de los clústeres aumentará; no obstante cada clúster mantendrá su tamaño original.