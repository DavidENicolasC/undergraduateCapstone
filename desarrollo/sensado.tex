\subsection{Sensado}

La empresa requirió que se utilizara una ortesis de brazo donde se montara el sistema de medición de los sensores inerciales y la Raspberry Pi. La Figura \ref{fig:ortesis} muestra la ortesis de brazo utilizada.

\begin{figure}[htb]
	\centering
	\includegraphics[scale=0.15]{ortesis.jpg}
	\caption{Ortesis utilizada para el proyecto}
	\label{fig:ortesis}
\end{figure}

Se realizó la conexión entre los sensores MPU y la Raspberry Pi 3 B+. La Figura \ref{fig:diagrama} muestra el diagrama de conexión entre los sensores y la Raspberry Pi. Los pines SDA y SCL de los sensores MPU-6050 se conectaron al bus serial $I^2C$, mientras que los pines MISO, MOSI, SCK y SS del sensor MPU-9250 se conectaron al bus serial SPI.

\begin{figure}[htb]
	\centering
	\includegraphics[scale=0.6]{diagrama.png}
	\caption{Diagrama de interconexión entre la Raspberry Pi y los sensores}
	\label{fig:diagrama}
\end{figure}

Nótese que tanto las líneas de datos (SDA), como las líneas de reloj (SCL) de los sensores MPU-6050, se conectaron a una única línea SDA o SCL, respectivamente, hacia la Raspberry Pi. Para evitar problemas de sincronización, se eligió la misma frecuencia de reloj para ambos sensores. Nótese también que se alimentó al sensor a través de la entrada AD0, en vez de VCC. Esto permite que su dirección en el bus $I^2C$ cambie de 104 a 105. Todos los sensores se alimentaron a través de la salida de voltaje de 3,3 V de la Raspberry Pi.

En lo que resta del documento, cuando se haga referencia a los sensores MPU-6050 y MPU-9250 en conjunto, se utilizará el prefijo MPU; cuando se haga referencia solo a uno de ellos, se hará con su nombre completo.

\subsubsection{Calibración}

Los sensores inerciales fabricados con tecnología MEMS, como los MPU, necesitan ser calibrados y tener un valor de referencia (offset) con el cual se corrija la orientación medida por el sensor \cite{offsetMPU}. El diagrama de la Figura \ref{fig:salidaMPU} obtenido del manual \cite{offsetMPU} muestra el proceso para obtener datos de los sensores MPU; los valores de referencia del acelerómetro y el giroscopio (Gyro and Accel Offset Registers) se aplican a las mediciones obtenidas por el giroscopio y el acelerómetro (Gyro and Accel MEMS), antes de colocar dichos valores en los registros del sensor (Gyro/Accel Output Registers), o ser procesados por el Procesador de Movimiento Digital (DMP) y el buffer FIFO.

\begin{figure}[htb]
	\centering
	\includegraphics[scale=0.8]{salidaMPU.png}
	\caption{Lectura de datos de los sensores MPU}
	\label{fig:salidaMPU}
\end{figure}

La Figura \ref{fig:ejes} muestra los ejes de desplazamiento y la rotación de los sensores MPU. Se colocaron los sensores MPU en la ortesis de modo que el sentido positivo del eje X quedara hacia el frente del operador (quien tiene colocada la ortesis); de acuerdo con esto, el sentido positivo de rotación en el eje Z se obtiene girando el brazo hacia la izquierda del operador; el sentido positivo de rotación en el eje Y se obtiene girando el brazo hacia abajo; y el sentido positivo de rotación en el eje X se obtiene rotando el brazo hacia la derecha.

\begin{figure}[htb]
	\centering
	\includegraphics[scale=0.5]{ejes.jpeg}
	\caption{Orientación de los ejes y polaridad de rotación de los sensores MPU}
	\label{fig:ejes}
\end{figure}

Se eligieron los valores de referencia de modo que la salida del sensor en el inicio del proceso de medir la orientación fuera $X=0, Y=0, Z=0$ para el giroscopio y $X=0, Y=0, Z=-9.8$ para el acelerómetro, debido a la constante de la aceleración de la gravedad g = -9.8  $m/s^2$. De acuerdo con lo anterior, al procesar los datos por el DMP, la salida deseada en el inicio del proceso de medir la orientación sería $\theta_x = 0°,\theta_y = 0°,\theta_z = 0°$.

Si el sensor se encuentra en estado de reposo (no existe ninguna fuerza externa que lo mueva), se espera que al leer datos de él, los valores no cambien; en la práctica, estos valores pueden variar debido a interferencias como el ruido externo. De modo que se calcula un valor medio cuyo error (la diferencia entre la medición del sensor en estado de reposo y el valor medio) sea menor que un error máximo aceptable; con base en la experiencia, se eligió un error máximo de 0.1 $\degree/s$ para el giroscopio, y 0.1 $m/s$ para el acelerómetro.

Para obtener dicho valor medio, se utilizó un control proporcional-integral (PI), en el cual se escribe el valor medido por el sensor en los registros de referencia (offset registers), y se compara dicho valor con la siguiente medición del sensor (con el último offset escrito en los registros de referencia aplicado a la nueva medición), para determinar el error; este proceso termina cuando el error obtenido se encuentra dentro del rango previamente establecido.

La Figura \ref{fig:calibracion} muestra el proceso de calibración del sensor. A cada medición se le aplicó el control PI para corregir el error. Para permitir que se establezca un valor apropiado, este proceso se repite 600 veces, un valor elegido basado en la experiencia. Después de que termina el ciclo, se compara el siguiente valor del sensor con los valores de referencia en los registros para determinar el error. Si éste es mayor que el error máximo aceptado, se repite el proceso.

Para que la calibración sea adecuada y se obtenga una medición confiable, el sensor debe de encontrarse en el estado de reposo mencionado anteriormente.

\begin{figure}[htb]
	\centering
	\includegraphics[scale=0.9]{calibracion.png}
	\caption{Diagrama a bloques del proceso de calibración del sensor}
	\label{fig:calibracion}
\end{figure}

Se utilizó un patrón de arquitectura de segmentación de procesos (process pipeline), en el cual los datos obtenidos del DMP se colocan en un buffer interno del sensor de tipo FIFO (El primer dato que entra, es el último que sale), del cual se lee la orientación del sensor; esto permite que los datos de la orientación puedan ser procesados sin que existan errores debido a lecturas incompletas. Aunque no se indica explícitamente la estructura de datos interna del buffer, se puede representar como un buffer compartido, esto es, una cola circular. La Figura \ref{fig:buffer} muestra el buffer compartido. Puede notarse que se modela de tal manera que el proceso productor (el DMP) y el proceso consumidor (la lectura de datos del sensor) no accedan al mismo valor.

\begin{figure}[htb]
	\centering
	\includegraphics[scale=0.9]{buffer.png}
	\caption{Representación del buffer interno del MPU como un buffer compartido}
	\label{fig:buffer}
\end{figure}

Posteriormente, se debe de cargar el firmware del procesador \cite{userguideMotionDriver}. Este proceso debe de realizarse cada vez que se encienda el sensor. Aunque no se indique explícitamente, el hecho de que el firmware deba de ser cargado cada vez que se inicie el sensor sugiere que la memoria que contiene el firmware del sensor es una memoria volátil. La memoria está formada por 8 bancos, en los que se carga el firmware proporcionado por InvenSense \cite{userguideMotionDriver}. 

Después de esto, se habilita el DMP y el buffer FIFO escribiendo el valor 1 en el bit FIFO\_EN (bit 6) del registro 106 indicados en las Figuras \ref{fig:fifoen} para la lectura de datos.

\begin{figure}[htb]
	\centering
	\includegraphics[scale=1]{fifoenable.png}
	\caption{Registro User Control del MPU, donde se encuentra el bit (Bit6) para activar el buffer FIFO}
	\label{fig:fifoen}
\end{figure}

\subsubsection{Medición}

El algoritmo utilizado por el DMP para obtener la orientación, no es de dominio público; en el Capítulo \ref{cap:anexos}, se describe el posible algoritmo utilizado por el DMP. Por otro lado, InvenSense, el desarrollador del sensor, ofrece un conjunto de bibliotecas para trabajar con el DMP \cite{userguideMotionDriver}.

El DMP utiliza cuaterniones para representar internamente la orientación; Estos son rápidamente computables y evitan problemas que se producen al girar más de 90°, como el bloqueo del cardán.

El cuaternión \cite{cuaternion} es de la forma:.

\begin{equation}
	q = a + b\hat{i} + c\hat{j} + d\hat{k}
	\label{eq:eqcuaternion}
\end{equation}

La Figura muestra el proceso general para obtener la orientación del sensor. Primero, se obtiene el cuaternión medido por el DMP del buffer FIFO. Posteriormente, se obtiene el vector de gravedad; este permite calcular la aceleración de la gravedad que experimenta el sensor en los 3 ejes, para tomarlo como sesgo y corregir la medición. Por último, se convierte lo antes calculado a valores de Euler, en radianes, y finalmente, dichos valores son convertidos a su equivalente en grados, que son los que se utilizan para indicar la orientación del sensor.

\begin{figure}[htb]
	\centering
	\includegraphics[scale=1]{fifoenable.png}
	\caption{Determinación de la orientación del sensor}
	\label{fig:orientacion}
\end{figure}

El vector de gravedad se determina como lo indica la ecuación :

Para obtener la orientación de los sensores MPU en los tres ejes en radianes, se utilizan las ecuaciones :

Los valores antes obtenidos son convertidos a su equivalente en grados, como lo indican las ecuaciones : 