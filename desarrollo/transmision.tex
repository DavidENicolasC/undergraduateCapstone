\subsection{Transmisión}

Se desarrolló e implementó la comunicación entre el software SettDev y el PLC para transmitir los datos leídos desde el archivo. El PLC-Kaab de SEDPC se controla y comunica utilizando datagramas; éstos se envían por medio del protocolo UDP, y tienen un formato de trama específico, que es el que se muestra en la Figura \ref{fig:tramaPLC} obtenida de SEDPC. La trama comienza con un encabezado (Header) que siempre contiene el valor 8A, seguido de algunas indicaciones para el PLC, como el proceso Fuente del que provienen los datos, el proceso de Destino, e indicaciones internas (Internal Indications). Luego, se indica un comando (Command), un subcomando (Subcommand), la indicación de algún Error producido, y la longitud $n$ de los datos por enviar (Largo). Los siguientes $n$ bytes contienen la información a enviar (Data). Por último, se calcula un Checksum como medio de comprobación de errores; si se produce algún error, el PLC no reproduce el datagrama enviado. Se utilizan identificadores con valores numéricos para indicar la Fuente o Destino (el proceso) que envía o recibe los datos, así como el comando y subcomando por ejecutar.

\begin{figure}[htb]
	\centering
	\includegraphics[scale=0.4]{tramaPLC.jpg}
	\caption{Formato de trama utilizado por el PLC-Kaab}
	\label{fig:tramaPLC}
\end{figure}

Para enviar los datos al PLC-Kaab, se tomó como referencia  un módulo para el control de servomotores desarrollado por SEDPC que se encuentra instalado en el PLC-Kaab. La trama generada para enviar los ángulos de inclinación al PLC se diseñó considerando la trama de este módulo, en el que el campo de Datos contiene 1 byte para representar el número de motor a controlar, y 4 bytes en formato Little Endian que representan el ángulo enviado a dicho motor. Entre cada instrucción el sistema espera un segundo, para permitir que el brazo robótico procese las instrucciones.